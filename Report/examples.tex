The example below presents how a class and the inheritance are depicted in C programming language.
\begin{lstlisting}
class Rectangle{
  var width: Int;
  var height: Int;

  def area(): Int = {
    return width * height;
  }
}

class Square extends{
  var side: Int;
}
\end{lstlisting}
This Tool code will be pretty-printed in C like this:
\begin{lstlisting}
struct Rectangle{
  int width;
  int heigth;
  int (*area)(struct void*);
}

struct Square{
  int width;
  int heigth;
  int side;
  int (*area)(struct void*)
}

  int Rectangle_area(struct Rectangle* this) {
    return this->width * this->height;
  }
\end{lstlisting}
The class are represented in C by a structure. The variables of the class
are members of the structure. They are declared in the same fashion.
\newline
Contrariwise the method of the class is declared outside of the structure.
In C, a structure can't have a function declaration as a member. The function shall
be declared outside. But the structure has a pointer to the function.
\newline
The function is declared with the name of the class as prefix followed by an underscore.
Thus the belonging to the \say{Rectangle} class is suggested.
\newline
Therefore this is how the C code mimics the behavior of an object-oriented class.
\newline
There exists other dilemma like inheritance, dynamic dispatch,\ldots, which be discussed
in the implementation section.
\newline
