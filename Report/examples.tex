The example below presents how a class and the inheritance could be depicted in C programming language.
\lstset{style=customtool}
\begin{lstlisting}
class Animal{
  var isAPet: Bool;
  var speed: Int;
  def sleep(): String = {
    return "zzz...";
  }
}

class Dog extends Animal{
  def bark(): String = {
    return "Woof!";
  }

  override def sleep(): String = {
    return "zzz... wouaf... zzz...";
  }
}
\end{lstlisting}
This Tool code will be pretty-printed in C like this:
\lstset{style=customc}
\begin{lstlisting}
struct Animal{
	int isAPet;
	int speed;
  char * (*sleep)(void *);
};

struct Dog{
	int isAPet;
	int speed;
  char * (*sleep)(void *);
	char * (*bark)(void *);
};

char * Animal_sleep (void * this) {
  // some code that will be depicted in the implementation part
}

char * Dog_sleep (void * this) {
  // some code that will be depicted in the implementation part
}

char * Dog_bark (void * this) {
  // some code that will be depicted in the implementation part
}
\end{lstlisting}
The class are represented in C by a structure. The variables of the class
are members of the structure. They are declared in the same fashion.
\newline
The variables of the class \say{Animal}, they also find themselves in the \say{Dog} structure
because the class \say{Dog} inherits from the class \say{Animal}.
\newline
Contrariwise the method of the class is declared outside of the structure.
In C, a structure can't have a function declaration as a member. The function shall
be declared outside. But the structure has a pointer to the function.
\newline
The function is declared with the name of the class as prefix followed by an underscore.
Thus the belonging to the \say{Dog} class is suggested.
\newline
Therefore this is how the C code mimics the behavior of an object-oriented class.
\newline
There exists other dilemma like method chaining, dynamic dispatch,\ldots, which be discussed
in the implementation section.
\newline
