The example below presents how a class and the inheritance could be depicted in the C programming language.
\lstset{style=customtool}
\begin{lstlisting}
class Animal{
  var isAPet: Bool;
  var speed: Int;
  def sleep(): String = {
    return "zzz...";
  }
}

class Dog extends Animal{
  def bark(): String = {
    return "Woof!";
  }

  //override
  def sleep(): String = {
    return "zzz... wouaf... zzz...";
  }
}
\end{lstlisting}
This Tool code will be pretty-printed in C like this:
\lstset{style=customc}
\begin{lstlisting}
struct Animal{
	int isAPet;
	int speed;
  char * (*sleep)(void *);
};

struct Dog{
  int isAPet;
  int speed;
  char * (*sleep)(void *);
  char * (*bark)(void *);
};

char * Animal_sleep (void * this) {
  // some code that will be depicted in the implementation part
}

char * Dog_sleep (void * this) {
  // some code that will be depicted in the implementation part
}

char * Dog_bark (void * this) {
  // some code that will be depicted in the implementation part
}
\end{lstlisting}
Classes are represented in C by a structure. The variables of the class
are members of the structure. They are declared in the same fashion.
\newline
The variables of the class \say{Animal}, are also located in the \say{Dog} structure
because the class \say{Dog} inherits from the class \say{Animal}.
\newline
Contrariwise to Tool, the methods of the class are declared outside of the structure.
In C, a structure can't have a function declaration as a member, thus the functions should
be declared outside, and the structure holds a pointer to the function.
\newline
The function is declared with the name of the class as a prefix followed by an underscore.
Thus the belonging to the \say{Dog} class is suggested, and this allows overriding (a same function with two implementations according to the class).
\newline
Therefore this is how the C code mimics the behavior of an object-oriented class.
\newline
There exists other dilemmas like method chaining, dynamic dispatch,\ldots, which will be discussed
in the implementation section.
\newline
